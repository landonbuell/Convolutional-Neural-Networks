% ================
% Landon Buell
% Qioayan Yu
% 
% 24 June 2020
% ================

\documentclass[12pt,letterpaper]{article}

\usepackage{float}
\usepackage{graphicx}
\usepackage{subfigure}
\usepackage{amsmath}
\usepackage{amssymb}
\usepackage[left=2.5cm,right=2.5cm,top=2.5cm]{geometry}

\begin{document}


% ================================================================

\section*{Experiment}

\paragraph*{}This case study serves as an experiment in determining how approximate computation techniques may change the performance of a multilayer perceptron (MLP) neural network classifier. We have chosen to a subsection of the Fashion-MNIST data set containing 28 x 28 pixel images of fashion accessories including handbags, shirts, hats, and shoes. Each image is labeled $0$ through $9$, encoding the article that appears with in. Each pixel is given by an integer $0$ to $255$ (a byte) which encodes t it's grey-scale value. In most cases, the subject extends to the outermost pixels of the image, which makes this data set favorable to the hand-written-digits-MNIST data set. The full data set was used which contains $60,000$ training samples and $10,000$ testing samples.

\paragraph*{}To apply an approximate computing technique to each image, we use a \textit{mute-bits} function. In doing so, each approximated pixel has gone from being stored as a specific $8$-bit object to $8$-bits of all $0$'s. This approximation method was applied to exterior pixels of each image sample, to effectively create a border of $N$-approximated pixels in depth. Models with $N = 2,4,6,8,10$ were compared to an unaffected baseline model. Examples of this approximation can be seen in fig. (\ref{images}).


\begin{figure}[h]
\label{images}
IMAGES!
\end{figure}


\paragraph*{}As a way to compensate for the previously described approximate computing technique, we also explore a method of attempting to recover the lost pixel border. We do this by using un-perturberbed pixels in the center of the image and copy and pasting them to the outside of the image. In doing this, the previous $N$-pixel border of muted bits is not replaced with features preserved in the center of the image. 




% ================================================================

\begin{figure}[H]
\label{results}
RESULTS!
\end{figure}

\section*{Conclusion}To mean any changes in performances of the neural-network classifier, we track the average validation loss value, and precision and recall scores




% ================================================================

\end{document}